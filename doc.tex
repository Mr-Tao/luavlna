\documentclass[12pt]{ltxdoc}
%\documentclass{article}
%\usepackage[utf8]{luainputenc}
\usepackage{fontspec}
\setmainfont[Ligatures=TeX]{Linux Libertine O}
\usepackage[czech,english]{babel}
\usepackage{luacode}
%\usepackage[]{polyglossia}
%\setmainlanguage{english}
%\setotherlanguage{czech}
%\usepackage{url}
\usepackage{hyperref}
%\input prevent-single
%\usepackage[]{prevent-single}
\usepackage[]{luavlna}
%\input prevent-single
%\def\preventsingledebugon{}
\newenvironment{mycode}{
	%\leavevmode%
	\medskip%
	\parindent=0pt%
}{\medskip}
\begin{document}
\title{The \verb|luavlna| package}
\author{Michal Hoftich (\url{michal.h21@gmail.com})}
\maketitle
\tableofcontents
\section{Introduction}


This is a small package for plain lua\TeX and lua\LaTeX. In some languages, like
Czech or Polish, there should be no single letter words at the line
end, according to the typographical norms. There exists some
external commands (like \verb!vlna!) or packages (\verb!encxvlna!
for enc\TeX, \verb!xevlna! for Xe\TeX,
\verb!impnattypo! for lua\LaTeX). %This package is for plain
%lua\TeX and for lua\LaTeX.

Other feature of this package is including of non-breakable space after 
initials, like in personal names. 

The code is modified version of Patrick Gundlach's answer on
TeX.sx\footnote{\url{http://tex.stackexchange.com/a/28128/2891}}.
The difference is that it is possible to specify which single letters
should be taken into account for different languages.
% The code works also for single letters at the beginning of the brackets.


\section{Usage}

The usage is simple:

\begin{verbatim}
\input ucode
\uselanguage{czech}
%% in the case of luacsplain, use instead:
%% \chyph
%% but language code for Czech is different than in LaTeX or normal 
%% luaTeX, so you will need to set single letters with somethinh like:
%% \singlechars{5}{AIiVvOoUuSsZzKk}
\input luavlna
\preventsingledebugon
\input luaotfload.sty
\font\hello={name:Linux Libertine O:+rlig;+clig;+liga;+tlig} at 12pt 
\hsize=3in
\hello
Příliš žluťoučký kůň úpěl ďábelské ódy. 
Text s krátkými souhláskami a samohláskami i dalšími jevy 
z nabídky možností (v textu možnými). 

I začátek odstavce je třeba řešit, i když výskyt zalomení není pravděpodobný.

Co třeba í znaky š diakritikou?

Různé možnosti [v závorkách <i jiných znacích

Podpora iniciál a titulů: M. J. Hegel, Ing. Běháková, Ph.D., Ž. Zíbrt.

Podpora jednotek: 60 m, 100,5 ns, 100.5 kJ, 2 µA, 8 MeV, $-1$ dag, 12 MiB.

\preventsingledebugoff
\bye
\end{verbatim}

%\noindent\parbox{3in}{%

\noindent
\begin{minipage}{3in}
\preventsingledebugon
\selectlanguage{czech}
Příliš \textit{žluťoučký kůň} úpěl ďábelské ódy. 
Text s krátkými souhláskami a samohláskami i dalšími jevy z nabídky možností (v textu možnými). 

I začátek odstavce je třeba řešit, i když výskyt zalomení není pravděpodobný.

Co třeba í znaky š diakritikou?

Různé možnosti [v závorkách \textless i jiných znacích

Podpora iniciál a titulů: M. J. Hegel, Ing. Běháková, Ph.D., Ž. Zíbrt.

Podpora jednotek: 60 m, 100,5 ns, 100.5 kJ, 2 µA, 8 MeV, $-1$ dag, 12 MiB.

\preventsingledebugoff
%}
\end{minipage}

\selectlanguage{english}
\bigskip
It is also possible to use the package with lua\LaTeX, just use

\begin{verbatim}
	\usepackage{luavlna}
\end{verbatim}

in the preamble.

\section{Commands}

\begin{mycode}
\cmd{\singlechars}\marg{language name}\marg{letters} 
\end{mycode}

Enable this feature for certain letters in given language. 
%Language code is internall \TeX\ code for the language, it is $0$ for English, 
%$16$ for Czech. Please note that in \verb|csplain|, language code for Czech 
%is $5$ and you will have to set it yourself.

Default values:

\begin{mycode}
\begin{verbatim}
%% only Czech and Slovak are supported out of the box
\singlechars{czech}{AIiVvOoUuSsZzKk}
\singlechars{slovak}{AIiVvOoUuSsZzKk}
\end{verbatim}
\end{mycode}

%By default, all
%single letters are processed, this command can be used to pass a
%string of characters, which should be processed only.

\begin{mycode}
\cmd{\compoundinitials}\marg{language name}\marg{compounds}
\end{mycode}

Declare compound letters for given language. Second argument should be comma 
separated list of compound letters, in exact form in which they can appear.

Default values:

\begin{mycode}
\begin{verbatim}
\compoundinitials{czech}{Ch,CH}
\end{verbatim}
\end{mycode}

\subsection{Turning off language switching}

By default, language of the nodes is taken into account. If you want to use
settings for one language for a whole document, you can use following command:

\begin{mycode}
\cmd{\preventsinglelang}\marg{language name}
\end{mycode}

\subsection{Turning off processing}

If you want to stop processing of the spaces in the document you can use command

\begin{mycode}
\cmd{\preventsingleoff}
\end{mycode}

To resume processing, use

\begin{mycode}
\cmd{\preventsingleon}
\end{mycode}

\subsection{Debugging commands}
\begin{mycode}
\cmd{\preventsingledebugon}\par
\cmd{\preventsingledebugoff}
\end{mycode}

Insert debugging marks on/off. Default off.

\section{Lua module \texttt{langno.lua}}

When we process glyph nodes with lua\TeX\ callbacks, there are \verb|lang| 
fields available. These are numerical codes of languages, but no information
about language names easily accesible from lua side is available.\footnote{%
	Language names are stored in \TeX macros like \verb|\string\l@langname|, but 
	different formats use different naming of these macros}
These numbers are format dependent, majority of formats like 
lua\LaTeX use \verb|language.dat| file provided by \verb|babel| 
for assign numbers to languages, but for example \verb|csplain| 
use its own system.

To allow easy setting of language dependent parameters, \verb|langno| module 
was created. It's purpose is to translate language code to language name and 
the other way. lua\LaTeX, lua\TeX\ and CSplain formats are supported at the moment.

\subsection{Recognized languages}
\subsubsection{lua\TeX\ and lua\LaTeX}

File \verb|language.dat| is processed to load language names, aliases and assigned numbers. These language names are the same as supported by \verb|babel| package.

\begin{quotation}
  \small\noindent
\begin{luacode*}
  langno = require "langno"
  function print_format_lang(fmt)
    local luatex = langno.load_languages(fmt)
    local t = {}
    for k, v in pairs(luatex.names) do
      t[#t+1]=k
    end
    table.sort(t)
    tex.print(table.concat(t,", "))
  end
  print_format_lang("luatex")
\end{luacode*}
\end{quotation}

\subsubsection{CSplain}

Different method is used. File \verb|hyphen.lan| is included in CSplain, where 
language numbers are assigned to ISO-639-1 or ISO-639-2 language codes. 
These language codes were then normalized to names used vy \verb|babel|, or 
standard English language names.

\begin{quotation}
  \small\noindent
  \begin{luacode*}
    print_format_lang("csplain")
  \end{luacode*}
\end{quotation}
\end{document}
